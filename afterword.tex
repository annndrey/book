%\footnotesize
\begin{center}
\subsubsection*{Послесловие от художника}
\addcontentsline{toc}{section}{Андрей Казило. Послесловие \newline художника}
\end{center}

Название этой, последней, книги Ольги Рожанской, <<Элизий земной>>, требует пояснений.
Почти все предыдущие сборники стихов Ольги Рожанской так или иначе аранжировали тему дороги, начиная с самого первого: <<Далее везде>>, <<Дорога в город>>.
Мотив дороги, катящегося колеса, обыгрывался в иллюстрациях к стихотворениям по-разному: один из вариантов обложки <<Carmen s$\ae$culare>> представлял (по просьбе автора) крылатые тележные колёса, врезавшиеся в здания Нью-Йоркских ``Близнецов''.

Кроме образа пути в творчестве Ольги Рожанской естественно присутствует и образ Града. Небесного Града... Его-то она всю жизнь взыскала! 
Дорога всегда заканчивается Градом -- хотя бы <<Уруком  ограждённым>> Гильгамеша, оплотом цивилизации в болотном хаосе, а уж для христианина целью пути является Вертоград Райский -- Небесный Иерусалим. Поэтому и на обложках сборников старался я Град, так или иначе, по мере сил, изобразить. 

В названии последнего сборника, кроме обычной для автора темы Вертограда, присутствует и иной аспект: будучи достаточно прохладным к творчеству Е.А. Баратынского, тем не менее я выделял, 
(кроме ещё нескольких любимых) -- стихотворение <<Пироскаф>>. 
В нем -- всегда привлекает некий парадокс. Последние строки:\\
%\newline
%% ****************  
%\settowidth{\versewidth}{Завтра увижу стены Ливурны!}
%\begin{verse}[\versewidth]
Завтра увижу стены Ливурны! \textbackslash{} Завтра увижу Элизий земной!\\
%\end{verse}
% ****************
обычно сопровождаются сноской: 
``апрель 1844 г. -- последнее стихотворение Баратынского... ночью, на пароходе... написал стихотворение <<Пироскаф>>\ldots 29 июня (11 июля) Евгений Баратынский скоропостижно скончался в Неаполе. Гроб с телом поэта поставлен в лютеранской церкви до перевозки на родину\ldots''

Ольга Рожанская знала весь комплекс идей и образов, связанных с 
предложенным мной названием сборника (обговоривали мы эту тему неоднократно), и 
легко утвердила его за три дня до отъезда на вечный отдых. Книгу планировалось печатать по возвращении\ldots
Собственно про иллюстрации~--- от первоначальной идеи использовать в оформлении книги текст <<Пироскафа>> пришлось отказаться. 
В настоящем оформлении фоном изображений служит Кафизма 17 Псалтири (Псалом 118), читаемую по усопшим 
в Русской Православной Церкви, чадом которой Ольга была до последнего вздоха. 
Так что, если наскучит читать стихи -- можно просто помолиться. И мысленно присутствовать среди собравшихся на погребение Ольги Рожанской.

\begin{flushright}
Андрей Казило, 2012 г.
\end{flushright}
\clearpage

